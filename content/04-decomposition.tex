\lchapter[decomposition]{Functional Decomposition}

% 4. (25%) Choose one of the concepts generated in Q3.
% Develop two different decompositions of function down to Level 2 (7+/-2 at L1 and ~25-50 at L2).
% Hint: review the "two down, one up" decomposition approach from SA4 in EM.411).
% Which decomposition do you prefer, and why? [1 page]

% Rubric
% Q4: Completeness
% Excellent (5 to >4.0 pts): Completed 2 decompositions
% Very Good (4 to >2.0 pts): Completed only decompositions or there is a mix of function and form listed
% Good (2 to >0 pts): too few decompositions, not based on function
%
% Q4: Approach to Decomposition
% Excellent (10 to >8.0 pts): Decompositions are meaningfully different and maintain solution neutrality over several concepts
% Very Good (8 to >4.0 pts): Decompositions are mostly different but may lack solution-neutrality
% Good (4 to >0 pts): See comments from grader
%
% Q4: Justification
% Excellent (5 to >4.0 pts): Strong statement of preferred decomposition with rationale
% Very Good (4 to >2.0 pts): Provided statement of preferred decomposition
% Good (2 to >0 pts): No clear statement about preferred decomposition.
%
% Q4: Guiding Principles & Level 1 Modularization
% Excellent (5 to >4.0 pts): Explicit guiding principle to generate L1 modularization based on L2
% Very Good (4 to >2.0 pts): Guiding principle is present but weakly connected to Level 1 modularization
% Good (2 to >0 pts): No guiding principle or justification for L1 modularization

% Failure modes
% * Decompositions aren't meaningfully different from each other (they are too locked in on one view)
% * Decompositions use overly generic wording without being "solution neutral."
% * Incorrect classification of "function."
% * No clear statement of what decomposition the team prefers
% * No or insufficient explanation of the team chose the decomposition they chose
% * No explanation of how the Level 2 of the decomposition informed the Level 1 (2 down, 1 Up)

\subsubsection{Selecting a Concept}
Concept \textbf{C-06} was selected to explore decomposition because it represents the most architecturally integrated and complex configuration within the explored tradespace. As an extreme case, it is perceived as exhibiting the highest degree of functional coupling, distributed decision-making, and continuous interaction among processing, networking, and resource management. Analyzing this concept was therefore expected to reveal the greatest insight into architectural boundaries, decomposition strategies, and system responsibility partitioning.

\subsubsection{Initial Level I Decomposition Approach}

The initial functional decomposition was derived directly from the project Concept of Operations (CONOP), organizing system behavior according to mission execution flow. This view emphasizes how sensing data progresses through the system from tasking through delivery, reflecting the operational sequence defined in the CONOP.

\paragraph{Initial Level I Functional Structure:}
\vspace{-2mm}
\noindent
\textit{Mission Planning \& Tasking $\rightarrow$
Health \& Status Monitoring $\rightarrow$
Data Acquisition $\rightarrow$
Onboard Processing $\rightarrow$
Satellite Networking \& Routing $\rightarrow$
Downlinking}\\

The above decomposition mirrors the operational phases identified in the CONOP and preserves traceability between functional activities and mission execution timelines. Functions are grouped according to temporal handoffs \textit{(e.g., what comes next?)}. Interfaces naturally align with transitions between mission phases \textit{(e.g., acquisition to processing or routing to downlink)}, making this view particularly effective for understanding end-to-end performance flow and identifying operational bottlenecks. As a result, the CONOP-based decomposition provides strong insight into system behavior and execution sequencing, though architectural ownership boundaries remain less explicit and sometimes even awkward.
\todo{Need to add refs.}

Level II decompositions were then derived from Level I. A full  NEED REF Initial Level I DSM is provided in the Appendix NEED REF. Because the Level I decomposition was derived directly from the CONOP, the operational perspective influenced not only how Level II functions were grouped, but also how they were defined and named. Functions were framed in terms of mission activities and execution phases, which implicitly emphasized sequential behavior over persistent system responsibilities. This naming convention shaped how functional relationships were interpreted and subsequently how dependencies were represented within the DSM.

\subsubsection{Reevaluation Approach}

Because of the size and density of the Design Structure Matrix (DSM), manual inspection alone made it difficult to clearly identify coupling patterns and functional groupings, particularly when influenced by user bias introduced through the initial decomposition perspective. The team therefore leveraged this exercise as an opportunity to explore DSM analysis tooling to support clustering and visualization. The Design Structure Matrix Editor (DSM Editor) Version 2.1.0 was used to analyze dependency structure and identify emergent clusters within the functional architecture. Automated clustering enabled clearer observation of tightly coupled functional groupings, supporting comparison between the CONOP-based decomposition and alternative architectural partitionings. The tool was used as an analytical aid to reveal structural relationships rather than to prescribe architectural decisions.

\subsubsection{Revised Level I Decomposition}