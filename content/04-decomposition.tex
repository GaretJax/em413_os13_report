\lchapter[decomposition]{Functional Decomposition}

% 4. (25%) Choose one of the concepts generated in Q3.
% Develop two different decompositions of function down to Level 2 (7+/-2 at L1 and ~25-50 at L2).
% Hint: review the "two down, one up" decomposition approach from SA4 in EM.411).
% Which decomposition do you prefer, and why? [1 page]

% Rubric
% Q4: Completeness
% Excellent (5 to >4.0 pts): Completed 2 decompositions
% Very Good (4 to >2.0 pts): Completed only decompositions or there is a mix of function and form listed
% Good (2 to >0 pts): too few decompositions, not based on function
%
% Q4: Approach to Decomposition
% Excellent (10 to >8.0 pts): Decompositions are meaningfully different and maintain solution neutrality over several concepts
% Very Good (8 to >4.0 pts): Decompositions are mostly different but may lack solution-neutrality
% Good (4 to >0 pts): See comments from grader
%
% Q4: Justification
% Excellent (5 to >4.0 pts): Strong statement of preferred decomposition with rationale
% Very Good (4 to >2.0 pts): Provided statement of preferred decomposition
% Good (2 to >0 pts): No clear statement about preferred decomposition.
%
% Q4: Guiding Principles & Level 1 Modularization
% Excellent (5 to >4.0 pts): Explicit guiding principle to generate L1 modularization based on L2
% Very Good (4 to >2.0 pts): Guiding principle is present but weakly connected to Level 1 modularization
% Good (2 to >0 pts): No guiding principle or justification for L1 modularization

% Failure modes
% * Decompositions aren't meaningfully different from each other (they are too locked in on one view)
% * Decompositions use overly generic wording without being "solution neutral."
% * Incorrect classification of "function."
% * No clear statement of what decomposition the team prefers
% * No or insufficient explanation of the team chose the decomposition they chose
% * No explanation of how the Level 2 of the decomposition informed the Level 1 (2 down, 1 Up)
