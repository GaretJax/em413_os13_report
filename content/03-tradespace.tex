\lchapter[tradespace]{Tradespace and Metrics}

% 3. (30%) By combining different decision-options, create 5-10 concepts (you can create more, if desired).
% Choose the two most important metrics (often one type of performance and one type of cost) and create a tradespace (you can keep your cost and performance evaluation high level at this stage, as you will later build a more detailed evaluation model in SE).
% Explain why you chose these specific metrics to evaluate your concepts.
% These metrics can be the same or different as the ones you used for sensitivity in OS11 Illustrate the Utopia point and the Pareto Front on your tradespace.
% Anchor your metrics by grading against the three historical designs from Q2. [2 pages]

% Rubric
% Q3: Clarity of Tradespace
% Excellent (4 to >3.0 pts): Visual encoding of Utopia point, Pareto front, and reference anchor
% Very Good (3 to >1.5 pts): Inadequate visual representation of the Utopia point, Pareto front, or reference anchor
% Good (1.5 to >0 pts): Some elements (Utopia point, Pareto front, or reference anchor) are missing or unclear
%
% Q3: Diversity of Tradespace
% Excellent (5 to >4.0 pts): 10 or More Points in Tradespace
% Very Good (4 to >2.0 pts): 9 to 6 points in tradespace
% Good (2 to >0 pts): Fewer than 5 points in the tradespace.
%
% Q3: Metrics and Evaluation
% Excellent (10 to >8.0 pts): Metrics are clearly explained and well-connected to the architecture
% Very Good (8 to >4.0 pts): Metrics are connected to the architecture
% Good (4 to >0 pts): see comments from grader
%
% Q3: Tension and Decision-making
% Excellent (5 to >4.0 pts): Well-defined tensions between axes leading to insightful trade-offs
% Very Good (4 to >2.0 pts): Some tension exists between axes
% Good (2 to >0 pts): No clear tension between axes. See comments from grader
%
% Q3: Analysis of Tradespace
% Excellent (6 to >5.0 pts): In-depth analysis of the tradespace, providing clear insights on decision connections and Pareto front decisions.
% Very Good (5 to >3.0 pts): Some discussion with analysis of Pareto front decisions and their connections.
% Good (3 to >0 pts): see comments from grader

% Failure modes
% * Tradespace is missing any of the following:
%   - Utopia point
%   - Pareto front (either missing or not drawn correctly)
%   - Insufficient architectures (<5)
% * No tension between the axes of the tradespace (they are co-variant, showing essentially y=x)
% * Tradespace anchors a favorite (all other architectures are just +/- 1-2% difference in metric values)
% * Metrics evaluation not credible (e.g. complex system price less than $1M)
% * Metrics not evaluated for a past design
% * The two metrics chosen weren't explained (especially if it is an aggregated metric like "utility") and/or metrics don't seem to be reasonable for the context
% * No clearly explained method to generate the 5-10 architectures
% * No clear driving architectural decision emerges (either through color-coded points or otherwise)
% * Tradespace attempts to show too many architectural decisions and is unreadable : as a rule 1, maybe 2 decisions should be illustrated with colors and / or point shapes, per tradespace.
% * No discussion or analysis of the tradespace
