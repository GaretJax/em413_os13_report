\lchapter[tradespace]{Tradespace and Metrics}

% 3. (30%) By combining different decision-options, create 5-10 concepts (you can create more, if desired).
% Choose the two most important metrics (often one type of performance and one type of cost) and create a tradespace (you can keep your cost and performance evaluation high level at this stage, as you will later build a more detailed evaluation model in SE).
% Explain why you chose these specific metrics to evaluate your concepts.
% These metrics can be the same or different as the ones you used for sensitivity in OS11 Illustrate the Utopia point and the Pareto Front on your tradespace.
% Anchor your metrics by grading against the three historical designs from Q2. [2 pages]

% Rubric
% Q3: Clarity of Tradespace
% Excellent (4 to >3.0 pts): Visual encoding of Utopia point, Pareto front, and reference anchor
% Very Good (3 to >1.5 pts): Inadequate visual representation of the Utopia point, Pareto front, or reference anchor
% Good (1.5 to >0 pts): Some elements (Utopia point, Pareto front, or reference anchor) are missing or unclear
%
% Q3: Diversity of Tradespace
% Excellent (5 to >4.0 pts): 10 or More Points in Tradespace
% Very Good (4 to >2.0 pts): 9 to 6 points in tradespace
% Good (2 to >0 pts): Fewer than 5 points in the tradespace.
%
% Q3: Metrics and Evaluation
% Excellent (10 to >8.0 pts): Metrics are clearly explained and well-connected to the architecture
% Very Good (8 to >4.0 pts): Metrics are connected to the architecture
% Good (4 to >0 pts): see comments from grader
%
% Q3: Tension and Decision-making
% Excellent (5 to >4.0 pts): Well-defined tensions between axes leading to insightful trade-offs
% Very Good (4 to >2.0 pts): Some tension exists between axes
% Good (2 to >0 pts): No clear tension between axes. See comments from grader
%
% Q3: Analysis of Tradespace
% Excellent (6 to >5.0 pts): In-depth analysis of the tradespace, providing clear insights on decision connections and Pareto front decisions.
% Very Good (5 to >3.0 pts): Some discussion with analysis of Pareto front decisions and their connections.
% Good (3 to >0 pts): see comments from grader

% Failure modes
% * Tradespace is missing any of the following:
%   - Utopia point
%   - Pareto front (either missing or not drawn correctly)
%   - Insufficient architectures (<5)
% * No tension between the axes of the tradespace (they are co-variant, showing essentially y=x)
% * Tradespace anchors a favorite (all other architectures are just +/- 1-2% difference in metric values)
% * Metrics evaluation not credible (e.g. complex system price less than $1M)
% * Metrics not evaluated for a past design
% * The two metrics chosen weren't explained (especially if it is an aggregated metric like "utility") and/or metrics don't seem to be reasonable for the context
% * No clearly explained method to generate the 5-10 architectures
% * No clear driving architectural decision emerges (either through color-coded points or otherwise)
% * Tradespace attempts to show too many architectural decisions and is unreadable : as a rule 1, maybe 2 decisions should be illustrated with colors and / or point shapes, per tradespace.
% * No discussion or analysis of the tradespace
\subsubsection{Identifying Concepts}

Ten distinct concepts were generated by systematically varying combinations of valid open architectural decisions identified in ~\ref{tbl:decisions}. 

Concept development built outward from a validated reference architecture, establishing the Airbus A450 bent-pipe configuration \textit{(C-01)} as the representative baseline concept. Subsequent concepts were then created by modifying individual architectural dimensions to isolate the performance implications of specific decisions, such as introducing onboard target recognition, reducing data products, enabling relay paths, or incorporating intersatellite networking. Later concepts intentionally combine multiple reinforcing architectural changes to explore coupled regimes, including mesh networking, onboard compute with metadata products, externally hosted transport networks, and high-bandwidth optical communication architectures.

\begin{table}[h!]
\centering
\footnotesize
\renewcommand{\arraystretch}{1.3}
\caption{Concept Morphological Matrix}
\begin{tabular}{|p{0.7cm}|p{2.0cm}|p{1.3cm}|p{1.8cm}|p{1.3cm}|p{1.1cm}|p{1.5cm}|p{1.5cm}|p{1.5cm}|p{1.8cm}|}
\hline
\textbf{ID} & 
\textbf{Concept Name} & 
\textbf{AD-01 (ATR)} & 
\textbf{AD-02 (Path)} & 
\textbf{AD-03 (Product)} & 
\textbf{AD-04.1 (Uplink)} & 
\textbf{AD-04.2 (Downlink)} & 
\textbf{AD-04.3 (Inter-Sat)} & 
\textbf{AD-05 (Routing)} & 
\textbf{AD-06 (Ground)} \\
\hline
C-01 & Bent-Pipe Baseline & None & Direct to Ground & Full Image & S-Band & X-Band & None & Static & Commercial GSaaS \\
\hline
C-02 & Smart Baseline & Task-Triggered & Direct to Ground & Snippet & S-Band & X-Band & None & Static & Commercial GSaaS \\
\hline
C-03 & Static GEO Relay & Task-Triggered & LEO-GEO External Relay & Snippet & S-Band & None (Relay) & Ka-Band & Static & Commercial GSaaS \\
\hline
C-04 & RF Mesh Network & Task-Triggered & LEO-LEO Internal & Snippet & S-Band & X-Band & X-Band & Adaptive Mesh & Hybrid \\
\hline
C-05 &  RF Store \& Forward & Task-Triggered & LEO-LEO Internal & Metadata & S-Band & Ka-Band & X-Band & Store \& Forward & Hybrid \\
\hline
C-06 & Compute Mesh & Continuous & LEO-LEO Internal & Metadata & S-Band & X-Band & X-Band & Adaptive Mesh & Hybrid \\
\hline
C-07 & Hosted Mesh & Continuous & LEO-LEO 3rd Party & Metadata & S-Band & X-Band & X-Band & Adaptive Mesh & Commercial GSaaS \\
\hline
C-08 & High Fidelity Rapid & Continuous & LEO-LEO Internal & Full Image & S-Band & Optical & Optical & Adaptive Mesh & Hybrid \\
\hline
C-10 & Regional Geographic Area & Task-Triggered & Direct to Ground & Snippet & S-Band & X-Band & None & Static & Commercial GSaaS \\
\hline
C-12 & LEO-GEO Optical & Task Triggered & LEO-GEO 3rd Party Relay & Snippet & S-Band & Nine (Relay)  & Optical & Static & Commercial GSaaS \\
\hline

\end{tabular}
\end{table}

\subsubsection{Determining Evaluation Metrics}
Latency and lifecycle cost were selected because they represent the system's primary sources of architectural tension. For example, decisions that reduce latency, such as inter-satellite networking or onboarding processing, typically increase system complexity and cost. Evaluating concepts across these competing dimensions exposes the trade space that governs the architectural selection. 

\begin{table}[h!]
\centering
\footnotesize
\renewcommand{\arraystretch}{1.3}
\begin{tabularx}{\textwidth}{|p{.8cm}|p{1.8cm}|p{2cm}|X|p{5cm}|}
\hline
\textbf{ID} &
\textbf{Type} &
\textbf{Metric} &
\textbf{Description} &
\textbf{Assessment Scale} \\
\hline

M1 &
Performance &
Latency &
Elapsed time from image acquisition to delivery of mission-ready data to the end user. &
5 = Low Latency |
1 = High Latency \\
\hline

M4 &
Cost &
Lifecycle Cost &
Total lifecycle cost benchmarked against comparable commercial Earth-observation systems. &
5 = Low Cost |
1 = High Cost \\
\hline

\end{tabularx}
\caption{Selected Evaluation Metrics}
\label{tab:eval_metrics}
\end{table}

Cost scoring reflects incremental architectural cost relative to the A450 bent-pipe baseline assuming funded commercial spacecraft, launch, ground services (e.g., KSAT/AWS), $\sim$200+ minutes/day X-band access, cloud-hosted CWAN POP, and funded GFE operations. Comparative cost differences across options informed the overall cost assessment.

\subsubsection{Evaluating Concepts}

\begin{table}[h!]
\centering
\footnotesize
\renewcommand{\arraystretch}{1.2}
\begin{tabularx}{\textwidth}{|p{2.6cm}|c|c|X|X|}
\hline
\textbf{Concept ID} & \textbf{Latency} & \textbf{Cost} & \textbf{Latency Rationale} & \textbf{Cost Rationale} \\
\hline

Utopia & 5 & 5 &
Ideal minimum-latency benchmark. &
Ideal minimum-cost benchmark. \\ \hline

C-01 & 1 & 4 &
Direct-to-ground delivery dominated by orbital access. &
Baseline A450 architecture with funded commercial ground. \\ \hline

C-02 & 2 & 4 &
ATR snippet reduces processing delay but remains contact-limited. &
Adds onboard processing with minimal hardware change. \\ \hline

C-03 & 2 & 3 &
GEO relay reduces ground dependence but adds relay delay. &
External relay avoids constellation-wide upgrades. \\ \hline

C-04 & 3 & 3 &
LEO mesh routes data toward earlier ground access. &
RF ISLs and routing integration add moderate complexity. \\ \hline

C-05 & 4 & 3 &
Metadata products reduce volume and improve delivery timing. &
RF mesh added without optical hardware burden. \\ \hline

C-06 & 4 & 4 &
Continuous ATR and adaptive routing reduce wait-to-contact effects. &
Persistent onboard compute and networking increase lifecycle cost. \\ \hline

C-07 & 4 & 2 &
Hosted mesh transport provides routing benefits externally. &
Transport infrastructure externalized via service model. \\ \hline

C-08 & 5 & 1 &
Optical links minimize transmission and routing bottlenecks. &
Optical terminals and precision pointing drive highest cost. \\ \hline

C-10 & 3 & 5 &
Regional geometry improves revisit but retains direct downlink limits. &
Specialized constellation reduces economies of scale. \\ \hline

C-12 & 3 & 3 &
Optical GEO relay improves access without full mesh networking. &
Adds relay terminal complexity but avoids constellation ISLs. \\ \hline

R-01 \textit{(OneWeb)} & 2 & 3 &
Bent-pipe commercial delivery with limited routing control. &
Moderate subscription-based transport cost. \\ \hline

R-02 \textit{(Iridium)} & 3 & 2 &
Operational RF mesh enables multi-hop gateway access. &
Mature external mesh minimizes mission-owned infrastructure. \\ \hline

R-03 \textit{(Starlink)} & 5 & 1 &
Dense constellation and crosslinks enable near-persistent connectivity. &
Highest infrastructure complexity to replicate or integrate. \\ \hline

\end{tabularx}
\caption{Concept Latency--Cost Evaluation}
\label{tab:concept_scores_rationale}
\end{table}

\subsubsection{Plotting the Tradespace}


ADD PLOT ADD PLOT ADD PLOT


The utopia point represents the hypothetical architecture achieving both minimum latency and minimum lifecycle cost simultaneously. While physically unattainable due to inherent performance–cost tradeoffs, it provides a reference against which concept efficiency can be evaluated. Given the ordinal scoring used in this study, the utopia point lies at the upper-right corner of the tradespace.
Decisions governing information location (AD-01, AD-03), transport topology (AD-02, AD-05), and communication modality (AD-04) determine whether latency is reduced through computation, networking, or transmission capacity. Concepts emphasizing onboard intelligence move vertically by reducing information entropy prior to transmission, while optical networking architectures achieve similar performance through increased bandwidth at higher lifecycle cost. Ground infrastructure and ownership models (AD-06) primarily influence cost positioning. Consequently, the Pareto frontier represents fundamentally different architectural philosophies rather than optimized variants of a single system design.

