\lchapter{Architectural Stability and Competition}

% Is the market for your system architecturally stable (e.g. many competitors all build the same architecture with minor design difference, like the A320 and B737), or is there substantial architectural competition (competitors build different architectures, such as the bag-style vacuum vs. the handheld Dyson vacuum vs. robotic “puck” vacuums)? Do you expect to come up with a brand new architecture, or a variation on an existing architecture / dominant design that is already sold in the market? (1 slide or 1 page ) [15%]

% Rubric
% Excellent (15 to >11.0 pts): Thoughtful and insightful response on whether the market is architecturally stable for the architecture that the team is currently considering, and shows research or evidence based on examples of other architectures in the selected industry and system. Strong argument on whether the team expects to generate a new architecture or variation on existing architecture. Synthesizes previous analyses on architectural decisions to suppor the team's response.
% Very Good (11 to >5.0 pts): Adequate response to the questions supported by some evidence from the team's architectural decisions analyses.
% Good (5 to >0 pts): Basic description of market trends and expectations. Inadequate evidence or data provided to support the team's response. See TA feedback.

% Failure modes
% * The degree of architectural change is overemphasized, or the definition of architecture is too narrow (e.g., asserting there is huge architectural competition in gasoline cars today)
% * Argument is contradictory (e.g., asserting there is both a dominant design and architectural change/competition, asserting market is architecturally stable but expecting to develop a new architecture without a good reason)
% * Business competition (e.g., on price) is confused with architectural competition
