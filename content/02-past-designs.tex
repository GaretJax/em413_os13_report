\lchapter[past-designs]{Past Design Solutions}

% 2. (10%) Show three past design solutions (systems that were built and sold) for your problem space, and which options they picked for your architectural decisions.
% Include, if applicable, the currently available for purchase solution that your system design might improve upon as one of the three.
% Illustrate each past design solution in the Q1 morphological matrix, to show how the 3 solutions differ based on your architectural decisions. [1 page]

% Rubric
% Q2: Framing Past Designs
% Excellent (4 to >3.0 pts): Clearly frames past designs using a structured list of architectural decisions.
% Very Good (3 to >2.0 pts): Discusses past designs but partially lacks architectural framing.
% Good (2 to >0 pts): Mentions past designs but weakly connects them to architectural decisions.
%
% Q2: Key Differences Between Past Designs
% Excellent (4 to >3.0 pts): Correctly distinguishes between design changes and major architectural shifts in the new proposed designs compared to a past design
% Very Good (3 to >2.0 pts): Occasionally misclassifies minor design changes as architectural.
% Good (2 to >0 pts): see comments from grader
%
% Q2: Proposed Improvements
% Excellent (2 to >1.0 pts): Clearly articulates major architectural differences between past designs and how proposed system could improve upon past systems
% Very Good (1 to >0.5 pts): Includes discussion about how proposed system could improve upon past designs, but does not cleary relate differences to identified architecture choices.
% Good (0.5 to >0 pts): See comments from grader

% Failure modes
% * No framing of past designs using a list of architectural decisions.
% * No reflection on the Product / System as a member of one class of architecture.
% * No articulation of how the future solutions may improve past designs.
% * Misclassification of small (design-level) changes between past designs as architectural.

To further contextualize the proposed architectural decisions, in this section we present an analysis of three past solutions, Iridium, OneWeb, and Starlink.

\subsubsection{Iridium (First Gen)}

Launched in the 1990s, Iridium was the first to implement some of the current considered architectural levers to bypass terrestrial limits, making them the pioneer in using inter-satellite networking \cite{gunOverviewIRIDIUMLow1998}.
\begin{itemize}[nosep,noitemsep]
  \item AD-02 (Data Path): LEO-LEO Internal Relay Mesh (fully meshed network): This was the first fully meshed commercial network where data was able to hop between satellites without needing a ground station for every pass.
  \item AD-04.3 (ISL): X-Band Radio Frequency: While optical links were considered, RF cross links were chosen because the tech was more mature at the time.
  \item AD-05 (Routing): Adaptive Dynamic Mesh: Research shows that early Iridium designs used static scheduling but was scrapped in favor of a dynamic approach to manage channel selection and routing late in the project
  \cite{IridiumSatelliteNetwork2022}.
  %\item AD-07 (Orbit): Near-polar LEO planes (Walker-Star pattern): Consisted of 66 active satellites across six polar planes at ~781 km altitude with 86.4 deg inclination.
\end{itemize}

%Iridium's architectural breakthrough was relocating routing responsibility from the ground segment to the space segment. This reduced terrestrial dependence and improved global continuity. Latency was improved structurally, though not optimized as a primary driver.  

\subsubsection{OneWeb (First Gen)}

These satellites were manufactured by Airbus (project sponsor) and provided the benchmark for mass-produced LEO platforms, making it an interesting legacy reference
\cite{OneWebMinisatelliteConstellation}.
\begin{itemize}[nosep,noitemsep]
  \item AD-02 (Data Path): Direct to Ground: Leverages a ``bent-pipe architecture'', no inter-satellite links, where signals are received in the Ka-band and relayed directly to users in the Ka-band.
  \item AD-04.1 (Uplink): Employed steerable Ka-band spot beams for ground gateway communications.
  \item AD-06 (Ground Segment): Relied on a global network of third-party and dedicated ground stations to handle traffic (Commercial GSaaS network of ground stations).
  %\item AD-07 (Orbit): Operates at 1,200 km with 648 satellites across 12 Polar orbital planes (LEO).
\end{itemize}

%OneWeb reflects a deliberate architectural simplification, one where routing and processing remain centralized in the ground segment. Unlike Iridium and StarLink, OneWeb does not distribute routing intelligence across the constellation. This is not merely a technology difference but a fundamentally different architectural commitment, prioritizing manufacturing scalability and gateway density over in-space network autonomy.

\subsubsection{SpaceX Starlink (Gen 1.5/2.0)}

This is the currently available solution that our project design aims to improve upon in terms of price, latency, and autonomous recognition
\cite{StarlinkPerformance2025}.
\begin{itemize}[nosep,noitemsep]
  \item AD-01 (ATR): Task-Triggered/Continuous (highly autonomous, indigenously-controlled constellation)
  \cite{StarlinkSatelliteConstellation}. Utilizes custom onboard AI (Stargaze) for autonomous collision avoidance and navigation.
  \item AD-04.3 (ISL): Optical Laser: SpaceX successfully transitioned from RF to Laser ISLs operating up to 200 Gbps.
  \item AD-05 (Routing): Adaptive Dynamic Mesh: Uses an autonomous global internet mesh to connect customers without requiring continuous reliance on a specific gateway.
  \item AD-06 (Ground Segment): Relied on a global network of self-owned and self-operated ground stations to handle traffic (Dedicated Network).
  %\item AD-07 (Orbit): Walker-Delta (Multi-plane) Operates a dense multi-plane muti-plane constellation at a lower altitude of ~550km to minimize signal latency.
\end{itemize}

\vspace{0.3\baselineskip}
\noindent While Starlink leads in network latency, our architecture has the ability to improve upon its Earth Observation (EO) capabilities by applying AD-01 (On-board Continuous ATR) and AD-03 (Information Product) to mission data.
This would allow for immediate intelligence delivery without the massive backhaul required for the ful-image downlinks required by continuous imaging.
Our project also aims to delivery similar latency to Starlink but at a vastly reduced cost while avoiding interface ``lock-in'' with proprietary SpaceX terminals.
Starlink advances the distributed LEO mesh architecture by integrating optical inter-satellite links and operating at lower orbital altitude to reduce propagation delay.

Looking across Iridium, OneWeb, and Starlink, the primary distinction is not hardware but the structural logic governing how information moves through the system.
Iridium first shifted routing authority into space using RF inter-satellite links, reducing dependence on ground visibility.
OneWeb made different architectural commitments, retaining routing and processing on the ground and treating satellites as bent-pipe relays to prioritize manufacturing scalability and gateway density.
Starlink pushed the distributed model further by embedding optical crosslinks and adaptive routing directly into the constellation, allowing the network itself to dynamically shape latency performance.
Across these systems, the relocation of routing authority (ground to space) represents a major architectural shift, whereas changes such as RF versus optical or orbital altitude are implementation decisions that reinforce that structural commitment.

To make these architectural choices explicit, each legacy system was mapped into the morphological matrix on \ref{tbl:decisions}.
By plotting their selected architectural decisions within the same structured framework, the similarities and divergences between the three systems become clearer.

Our proposed design reflects a deliberate shift in emphasis.
Rather than treating latency purely as a function of link technology, we treat the structure of information flow as the primary architectural lever.
Continuous onboard processing and selective data products reduce volume before it enters the network, shifting the problem from moving large files quickly to delivering the right information immediately.
By elevating routing structure as the architectural anchor, we avoid over-committing to any single ground infrastructure model or proprietary interface and instead re-balance the distributed mesh around mission-relevant timeliness rather than throughput.
What improves over past systems is not simply speed, but coherence: instead of optimizing components in isolation, the architecture aligns onboard processing, routing authority, and ground entry strategy around a unified performance objective. This structural alignment reduces unnecessary backhaul, limits dependence on terrestrial chokepoints, and preserves flexibility as mission demands evolve. The improvement shifts processing and routing authority, making it architectural rather than technological.
%The improvement is architectural before it is technological.
