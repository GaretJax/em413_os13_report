\lchapter[sps]{System Problem Statement}

% For your project, create or refine the system problem statement, and discuss why you chose to the breadth or narrowness that you did. Remember, your team gets to choose the scope for the project, not the sponsor. (0.5 slide or page) [5%]

% Rubric
% Excellent (5 to >4.0 pts): Clear and concise SPS that is well-balanced: neutral enough to allow search for a solution and focused enough to create meaningful direction. Insightful discussion on the level of SPS that was chosen.
% Very Good (4 to >2.0 pts): Solid SPS with some minor gaps in clarity, usefulness of level chosen, or justification.
% Good (2 to >0 pts): SPS framing, level of breadth / narrowness, and rationale exhibit common failure modes. See TA feedback.

% Failure modes
% * To statement is insufficiently solution-neutral for the problem
% * To statement expresses a goal that will not be meaningfully addressed by the architecture (e.g., “to solve world hunger by providing school lunches in Waterloo”)
% * To statement expresses a goal that may be difficult to judge achievement against or is subject to multiple interpretations (e.g., “to improve fun quality for children”)
% * To statement expresses a goal that is already present or too generic to be related to the project (e.g., “to increase profitability”, “to deliver products to customers”)
% * To statement uses generic language but appears to mask a solution-specific idea with jargon (e.g., “to assist with the extraction of contaminants and surface leveling of grime from dishes”)
% * By statement expresses a broad trend, not a process (e.g., “leveraging new technology”)
% * By statement is unnecessarily generic and does not describe an action that is separate from the to statement (e.g., “to scale operations by automating workflow”)
% * By statement does not meaningfully accomplish To statement (e.g., “to design a vehicle by evaluating customer needs”)
% * By statement uses linguistic crutches (e.g., “by enabling”, “by providing for”, “by ensuring”)
% * Using expresses the system architecture process or a method, not the potential form
% * Using expresses a broad idea that is not linked to the problem (e.g., “using AI”, “using a smart design team”, “using blockchain”)
% * Using simply re-expresses the sponsor’s technology choice or product, without the team’s input (e.g., “using Sony’s mirrorless camera technology”, “using the offshore platform”)
% * Using or by expresses the final solution, not a framing of the problem
% * To, by, using express emergent properties that are not within their control (e.g., “by producing a profitable product”, “by standardizing XYZ across the industry”, “using an efficient/innovative process”)
% * To, by, using don’t offer a meaningful search direction separate from the sponsor’s brief (e.g., “to reduce emissions by replacing the energy source of cars using electrification”)
% * To, by, using don’t hang together, or seem to focus in on a minor area of the problem, ignoring a big piece of the scope
% * To, by, using offer too much detail or produce a combined statement that is verbose, rife with too many adjectives, murky, or not crisp
