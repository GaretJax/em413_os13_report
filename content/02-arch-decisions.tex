\lchapter[decisions]{Architectural Decisions}

% Define a set of candidate architectural decisions for your system (no more than 10 decisions), with mutually exclusive options for each decision. Try to be as specific as possible to your problem. Make sure that each decision is technical, in the company’s control, and that you are showing technical choices to be made (1,2,3 backup generators), not inputs or scope choices (will it be used off-grid?), or outcomes (reliability of 99%). For each decision, identify whether it is a decision on form, function, or form to function mapping. (1 slide or 1 page) [25%]

% Rubric
% Excellent (25 to >18.0 pts): Identified a list of architectural decisions that encompass the scope of the intended system, defined mutually exclusive options for each decision, and properly classified each decision to form, function, or form-to-function mapping, with a balance of decisions that relate to form, function, and form-to-function mapping. Clearly shows that decisions are capable of generating many valid and alternative architectures. Provides rationale for the decisions, options, and classifications.
% Very Good (18 to >9.0 pts): Provided solid architectural decisions and descriptions with minor errors in terms of not mutually exclusive options, outcomes listed as decisions, or goals listed as decisions.
% Good (9 to >0 pts): Provided decisions that are not true technical architectural decisions or missing crucial decisions for the intended system. Improper options for each decision are listed or errors in decision classification. See TA feedback.

% Failure modes
% * Decisions are not “architectural” (e.g., "color", "leather/cloth", and "wheel diameter" for a car)
% * Decisions relate to a process, not the architecture (a set of technical design decisions, the "how" that accomplishes the requirements/goals)
% * Decisions are applied to non-technical aspects. NOTE: The word architecture is reserved for technical consideration, and excludes business process, organizational decisions, etc.
% * Almost all the decisions are about form (less likely to be architectural)
% * Very few or no decisions are about form-function mapping
% * Function decisions are mis-categorized as form decisions, or vice versa
% * Options are not mutually exclusive or are awkwardly framed so as to make them mutually exclusive
% * Architectural decisions are mixed with scope, scenarios, intent, goals, metrics, or requirements (e.g., choosing a price point for the system)
% * An outcome of the system (emergence) is confused with architectural decisions (e.g., “should it be easy to use or not”, “should it be efficient or not”)
% * More than 10 decisions are provided
% * Decisions are not sensitive or connected
% * Decisions are not within the architect’s control

\ref{tbl:decisions} presents the set of candidate architectural decisions identified for the satellite system under study. Because of this, it is important to note that these decisions are specific to the satellite bus transmission communications payload.
Each decision represents a technical choice within the system developer’s control, within the system boundary, and materially influences the system's structure or behavior. For each architectural decision, a discrete set of mutually exclusive options is defined, representing alternative ways the system could be architected. 

\begin{table}
  \caption{Architectural decisions.}
  \label{tbl:decisions}
  \footnotesize

\newcounter{rownumber}

\newcommand{\row}[9]{
  \stepcounter{rownumber}
  \textbf{#1}
  \textit{
    \scriptsize
    \ifstrequal{#2}{form}{Form}{}
    \ifstrequal{#2}{func}{Function}{}
    \ifstrequal{#2}{f2f}{F2F}{}
  }
  &
  #3 &
  #4 &
  #5 &
  #6 &
  \ifstrequal{#7}{-}{\cellcolor{black!8}\;}{#7} &
  {\scriptsize\StrSubstitute{#8}{,}{\newline}} &
  \ifstrequal{#9}{Open}{\cellcolor{Snow1}}{}
  \ifstrequal{#9}{Updated}{\cellcolor{SlateGray1}}{}
  \ifstrequal{#9}{New}{\cellcolor{Khaki1}}{}
  #9 \\
}

\makeatletter
\pgfdeclareplotmark{tnw}{
  \pgfpathmoveto{\pgfqpoint{-2pt}{2.5pt}}
  \pgfpathlineto{\pgfqpoint{4pt}{2.5pt}}
  \pgfpathlineto{\pgfqpoint{-2pt}{-3.5pt}}
  \pgfpathclose
  \pgfusepathqfillstroke
}
\pgfdeclareplotmark{tne}{
  \pgfpathmoveto{\pgfqpoint{2pt}{2.5pt}}
  \pgfpathlineto{\pgfqpoint{-4pt}{2.5pt}}
  \pgfpathlineto{\pgfqpoint{2pt}{-3.5pt}}
  \pgfpathclose
  \pgfusepathqfillstroke
}
\pgfdeclareplotmark{tsw}{
  \pgfpathmoveto{\pgfqpoint{-2pt}{-2.5pt}}
  \pgfpathlineto{\pgfqpoint{4pt}{-2.5pt}}
  \pgfpathlineto{\pgfqpoint{-2pt}{3.5pt}}
  \pgfpathclose
  \pgfusepathqfillstroke
}
\makeatother

\renewcommand{\arraystretch}{1.4}
\setlength{\tabcolsep}{4pt}
\definecolor{iridiumyellow}{HTML}{F3B207}
\definecolor{onewebblue}{HTML}{57BEF2}
%\renewcommand{\tabularxcolumn}[1]{m{#1}}

\begin{NiceTabularX}{\linewidth}{ %
    | >{\setstretch{1}\raggedright\arraybackslash}m{12mm} %
    | >{\setstretch{1}\raggedright\arraybackslash}m{32mm} %
    | *{4}{>{\centering\arraybackslash}X[m]|} %
      M{9mm} %
    | M{11mm} | %
  }[color-inside]
  \hline
    \Th{ID} &
    \Th{Architectural Decision} &
    \Th{Option A} &
    \Th{Option B} &
    \Th{Option C} &
    \Th{Option D} &
    \Th{Trac.} &
    \Th{Status} \\
  \hline
    \row{AD-01}{func}{On-board Autonomous Target Recognition (ATR)}{None}{Task-Triggered}{Continuous}{-}{PSF-3, FR-007}{Open}
  \hline
    \row{AD-02}{f2f}{Data Path}{Direct to Ground}{LEO-LEO Internal Relay Mesh}{LEO-LEO 3rd Party Relay Mesh}{LEO-GEO External Relay}{PSF-5, FR-005, FR-006}{Open}
  \hline
    \row{AD-03}{func}{Information Product}{Full Image Only}{Snippet/ Thumbnail}{Metadata}{-}{PSF-4}{Open}
  \hline
    \row{AD-04.1}{form}{Uplink Communication}{Optical Laser}{X-Band RF}{\sout{Ka-Band}

    S-Band RF}{-}{FR-005, FR-006, NFR-02}{Updated}
  \hline
    \row{AD-04.2}{form}{ Downlink Communication}{Optical Laser}{X-Band RF}{Ka-Band RF}{None (external Relay Downlink)}{FR-005, FR-006, NFR-02}{Open}
  \hline
    \row{AD-04.3}{form}{Inter-satellite Communication}{Optical Laser}{X-Band RF}{Ka-Band RF}{None}{FR-005, FR-006, NFR-02}{Open}
  \hline
    \row{AD-05}{f2f}{Data Routing Protocol}{Static Pre-Scheduled}{Adaptive Dynamic Mesh}{Store and Forward}{-}{PSF-5, NFR-01, NFR-02}{Open}
  \hline
    \row{AD-06}{form}{Ground Segment Network}{Dedicated Sovereign Network}{Commercial GSaaS (\eg KSAT/AWS)}{Hybrid (Sovereign + Commercial)}{-}{PSF-5, FR-006, NFR-01}{New}
  \hline
\CodeAfter
  \tikz[remember picture, overlay]{

    \node[draw=black, xshift=1.6mm, yshift=-5mm, color=black, inner sep=0, minimum width=26mm, minimum height=10mm, anchor=north east] (l) at (9-8.south east) {};

    \node[align=right,text width=42mm,anchor=north east] at (l.north west) {
      \RaggedLeft
      Legend for past design \\
      solutions described in \ref{sec:past-designs}:
    };

    % Iridium
    \draw[
      iridiumyellow,
      only marks,
      yshift=0mm,
      line width=0.8pt,
      mark=tnw,
      mark options={solid},
    ] plot coordinates {
      ([yshift=0mm]2-3.north west) % None
      ([yshift=0mm]3-4.north west) % LEO-LEO
      ([xshift=0mm]4-6.north west) % n/a (voice)
      ([yshift=0.6em]5-6.north west) % n/a (Ka-Band)
      ([yshift=0.6em]6-5.north west) % Ka-Band
      ([yshift=0.6em]7-5.north west) % Ka-Band
      ([yshift=0mm]8-4.north west) % Mesh
      ([yshift=0mm]9-3.north west) % Dedicated
      ([xshift=1.6mm,yshift=-1.6mm]l.north west)
    };
    \node[yshift=0.2em,xshift=-0.1em,below right,iridiumyellow!70!Sienna4,font=\tiny] at (4-6.north west) {Data (voice)};
    \node[yshift=0.8em,xshift=-0.1em,below right,iridiumyellow!70!Sienna4,font=\tiny] at (5-6.north west) {Ka-Band RF};
    \node[yshift=-0.4em,xshift=0.4em,below right,iridiumyellow!70!Sienna4,font=\footnotesize] at (l.north west) {Iridium};

    % OneWeb
    \draw[
      onewebblue,
      only marks,
      yshift=0mm,
      line width=0.8pt,
      mark=tne,
      mark options={solid},
    ] plot coordinates {
      ([yshift=0mm]2-3.north east) % None
      ([yshift=0.6em]3-3.north east) % Direct to ground
      ([xshift=0mm]4-6.north east) % n/a (data)
      ([yshift=0.6em]5-6.north east) % Ka-Band
      ([yshift=0.6em]6-5.north east) % Ka-Band
      ([yshift=0.6em]7-6.north east) % None
      ([yshift=0mm]8-3.north east) % Fixed
      ([yshift=0mm]9-5.north east) % Mixed
      ([xshift=-1.6mm,yshift=-1.6mm]l.north east)
    };
    \node[yshift=0.2em,xshift=-0.1em,below left,onewebblue!70!Turquoise4,font=\tiny] at (4-6.north east) {Data (internet)};
    \node[yshift=0.8em,xshift=-0.1em,below left,onewebblue!70!Turquoise4,font=\tiny] at (5-6.north east) {Ka-Band RF};
    \node[yshift=-0.4em,xshift=-0.4em,below left,onewebblue!70!Turquoise4,font=\footnotesize] at (l.north east) {OneWeb};

    % Starlink
    \draw[
      red,
      yshift=0mm,
      only marks,
      line width=0.8pt,
      mark=tsw,
      mark options={solid},
    ] plot coordinates {
      ([yshift=0mm]2-3.south west) % None
      ([yshift=0mm]3-4.south west) % LEO-LEO
      ([yshift=0mm]4-6.south west) % n/a (data)
      ([yshift=-0.6em]5-6.south west) % Ka-Band
      ([yshift=-0.6em]6-5.south west) % Ka-Band
      ([yshift=-0.6em]7-3.south west) % Laser
      ([yshift=0mm]8-4.south west) % Mesh
      ([yshift=0mm]9-3.south west) % Dedicated
      ([xshift=1.6mm,yshift=1.6mm]l.south west)
    };
    \node[yshift=-0.3em,xshift=-0.1em,above right,red,font=\tiny] at (4-6.south west) {Data (internet)};
    \node[yshift=-0.8em,xshift=-0.1em,above right,red,font=\tiny] at (5-6.south west) {Ka-Band RF};
    \node[yshift=0.4em,xshift=0.4em,above right,red,font=\footnotesize] at (l.south west) {Starlink};

    % % Iridium
    % \draw[
    %   iridiumyellow,
    %   only marks,
    %   yshift=0mm,
    %   line width=0.8pt,
    %   mark=*,
    %   mark size=1.8pt,
    %   mark options={solid},
    % ] plot coordinates {
    %   ([yshift=0mm]2-3.south west) % None
    %   ([yshift=0mm]3-4.south west) % LEO-LEO
    %   ([xshift=4mm]4-6.south west) % n/a (voice)
    %   ([xshift=4mm]5-6.north west) % n/a (Ka-Band)
    %   ([yshift=0mm]6-5.north west) % Ka-Band
    %   ([yshift=0mm]7-5.north west) % Ka-Band
    %   ([yshift=-1mm]8-4.west) % Mesh
    %   ([yshift=0mm]9-3.north) % Dedicated
    % };
    % \node[xshift=4mm,right,iridiumyellow!70!Sienna4,font=\tiny] at (4-6.south west) {Data (voice)};
    % \node[xshift=4mm,right,iridiumyellow!70!Sienna4,font=\tiny] at (5-6.north west) {Ka-Band RF};

    % % OneWeb
    % \draw[
    %   onewebblue,
    %   only marks,
    %   yshift=0mm,
    %   line width=0.8pt,
    %   mark=*,
    %   mark size=1.8pt,
    %   mark options={solid},
    % ] plot coordinates {
    %   ([yshift=0mm]2-3.south east) % None
    %   ([yshift=1.8mm]3-3.east) % Direct to ground
    %   ([xshift=0mm]4-6.north west) % n/a (data)
    %   ([xshift=0mm]5-6.south west) % Ka-Band
    %   ([yshift=0mm]6-5.south east) % Ka-Band
    %   ([yshift=0mm]7-6.west) % None
    %   ([yshift=0mm]8-3.south east) % Fixed
    %   ([yshift=0mm]9-5.south west) % Mixed
    % };
    % \node[yshift=0mm,xshift=0mm,right,onewebblue!70!Turquoise4,font=\tiny] at (4-6.north west) {Data (internet)};
    % \node[xshift=0mm,right,onewebblue!70!Turquoise4,font=\tiny] at (5-6.south west) {Ka-Band RF};

    % % Starlink
    % \draw[
    %   red,
    %   yshift=0mm,
    %   only marks,
    %   line width=0.8pt,
    %   mark=*,
    %   mark size=1.8pt,
    %   mark options={solid},
    % ] plot coordinates {
    %   ([yshift=0mm]2-3.north east) % None
    %   ([yshift=0mm]3-4.north east) % LEO-LEO
    %   ([yshift=0mm]4-6.north east) % n/a (data)
    %   ([yshift=-1.8mm]5-6.south east) % Ka-Band
    %   ([yshift=0mm]6-5.north east) % Ka-Band
    %   ([yshift=0mm]7-3.east) % Laser
    %   ([yshift=0mm]8-4.north west) % Mesh
    %   ([yshift=0mm]9-3.north west) % Dedicated
    % };
    % \node[yshift=0mm,xshift=0mm,below left,red,font=\tiny] at (4-6.north east) {Data (internet)};
    % \node[yshift=-1.8mm,above left,red,font=\tiny] at (5-6.south east) {Ka-Band RF};


  }%
  \vspace{10mm}
\end{NiceTabularX}

\end{table}

To reinforce consistency and disciplined system design, a traceability column has been added to the architectural decisions table. This column explicitly links each architectural decision to the system objectives, primary system functions (PSFs), and functional (FRs) or non-functional requirements (NFRs) documented and presented in OS 9+10.

This traceability clarifies how architectural decisions emerge from validated stakeholder needs. In several cases, a single architectural decision supports multiple objectives or requirements, reflecting the system-level impact of architectural decisions. Conversely, some objectives and requirements influence architectural decisions indirectly by shaping constraints or performance sensitivities rather than mapping directly.

By making these relationships explicit, the traceability column demonstrates that each architectural decision is grounded in the system’s operational intent and performance goals, and that architectural tradeoffs explored in future subsequent analysis are directly tied to mission-relevant outcomes.

It is to be noted that through this architectural decision exploration space, the following architectural elements are externally constrained by the end user and therefore fall outside of AIRBUS’s architectural control for this study. These elements are treated as fixed system assumptions and are not included as candidate architectural decisions:

\begin{itemize}
  \item \textbf{Orbit:} The orbital regime is defined by the end user based on mission-specific data acquisition requirements.
  \item \textbf{Constellation Size:} The number of satellites procured is determined by the end user. A minimum constellation size of 20 satellites is assumed for this analysis.
  \item \textbf{Ground Segment:} Ground station selection and operation are controlled by the end user. For this study, a single ground station located at the North Pole is assumed (pending confirmation with AIRBUS).
  \item \textbf{Payload:} The mission payload is defined by the end user. An optical sensor payload is assumed for this study.
\end{itemize}
