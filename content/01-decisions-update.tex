\lchapter[decisions-update]{Updated Architectural Decisions}

% 1. (10%) Update your list of architectural decisions from OS11, identifying which decisions have already been made, which decisions remain, and any new decisions that may have been identified.
% Explain each and summarize why your final set of decisions satisfies the problem space (i.e., your SPS).
% Show your decisions as table (aka morphological matrix) [1 page]

% Rubric
% Q1: Updated Decisions: Architectural Focus
% Excellent (3 to >2.5 pts): All decisions are technical and architectural in nature (excluding process, form-based, or business-related decisions). There is a strong mapping between form and function, with no misclassified decisions.
% Very Good (2.5 to >1.5 pts): 1-2 minor process or form-based decisions are included or 1-2 architectural decisions are misclassified.
% Good (1.5 to >0 pts): More than 3 non-architectural decisions are present (e.g., related to process, business, aesthetics) or more than 3 architectural decisions are influenced by form.
%
% Q1: Updated Decisions: Justifications
% Excellent (3 to >2.5 pts): There must be a clear, explicit rationale explaining how existing decisions satisfy the SPS. Explanations for closed decisions are clear, and any remaining or newly identified decisions have appropriate rationale.
% Very Good (2.5 to >1.5 pts): Only some updates are explained, or does not cleary express how existing decisions satisfy the SPS.
% Good (1.5 to >0 pts): See comments from grader
%
% Q1: Updated Decisions: Trade-offs & Emergence
% Excellent (2 to >1.0 pts): All decisions represent mutually exclusive trade-offs focused on architecture rather than just system properties (like usability).
% Very Good (1 to >0.5 pts): Minor overlaps exist among decisions, with some references to emergent properties.
% Good (0.5 to >0 pts): Confuses system outcomes (e.g., usability) with architectural decisions. See comments from grader
%
% Q1: Updated Decisions: Morphological Matrix
% Excellent (2 to >1.0 pts): Decisions are clearly represented in morphological matrix.
% Very Good (1 to >0.5 pts): Morphological matrix is provided but contains ambiguity or incorrect formatting.
% Good (0.5 to >0 pts): see comments from grader

% Failure modes
% * Decisions are selected that no one would reasonably call "architectural" (e.g., color, leather, wheel diameter)
% * Decisions relate to a process, not the architecture.
% * Decisions applied to non-technical aspects.
% * Too many decisions focus on form; very few/no decisions on form-function mapping.
% * Misclassification of function decisions as form decisions or vice versa.
% * If architectural decisions were made since the list in OS11, no explanation of the rationale for those decisions.
% * Options are not mutually exclusive or awkwardly framed to make them appear so.
% * Confusing an outcome of the system (emergence) with architectural decisions (e.g., "should it be easy to use or not ").
% * Mixing architectural decisions with scope, scenarios, intent, goals, metrics, or requirements.
% * Decisions that aren't within the architect's control.
% * Decisions aren't sensitive or connected.
% * Too many decisions to possibly all be architectural in nature (e.g., more than 10).

\setlength{\fboxsep}{3mm}
\noindent\fbox{\parbox{\dimexpr\linewidth-2\fboxsep-2\fboxrule}{
  \textbf{TO} enable global decision making through persistent, low-latency space-based data \\
  \textbf{BY} dynamically routing data through a networked constellation to bypass terrestrial infrastructure limitations \\
  \textbf{USING} a networked pLEO constellation architecture.
}}
\vspace{2mm}

\noindent After sponsor discussions since OS11, we have moved Ground Segment from fixed assumptions to an architectural decision (AD-06) to better address end-to-end latency. This reframing clarified that latency is not solely a function of link bandwidth, but that ground stations density also directly impact downlinking opportunities. Additional feedback and discussion with our sponsor led us to update AD-04.1 (Uplink Comms) because Ka-Band was determined to introduce excessive costs without impact on latency when compared to already available S/X-band radios. The final architectural decision set, refined to better reflect the structural drivers required to satisfy our SPS, is illustrated in \ref{tbl:decisions}.

\begin{table}[h!]
  \caption{Morphological matrix, including past design solutions and updated architectural decisions.}
  \label{tbl:decisions}
  \footnotesize

\newcounter{rownumber}

\newcommand{\row}[9]{
  \stepcounter{rownumber}
  \textbf{#1}
  \textit{
    \scriptsize
    \ifstrequal{#2}{form}{Form}{}
    \ifstrequal{#2}{func}{Function}{}
    \ifstrequal{#2}{f2f}{F2F}{}
  }
  &
  #3 &
  #4 &
  #5 &
  #6 &
  \ifstrequal{#7}{-}{\cellcolor{black!8}\;}{#7} &
  {\scriptsize\StrSubstitute{#8}{,}{\newline}} &
  \ifstrequal{#9}{Open}{\cellcolor{Snow1}}{}
  \ifstrequal{#9}{Updated}{\cellcolor{SlateGray1}}{}
  \ifstrequal{#9}{New}{\cellcolor{Khaki1}}{}
  #9 \\
}

\makeatletter
\pgfdeclareplotmark{tnw}{
  \pgfpathmoveto{\pgfqpoint{-2pt}{2.5pt}}
  \pgfpathlineto{\pgfqpoint{4pt}{2.5pt}}
  \pgfpathlineto{\pgfqpoint{-2pt}{-3.5pt}}
  \pgfpathclose
  \pgfusepathqfillstroke
}
\pgfdeclareplotmark{tne}{
  \pgfpathmoveto{\pgfqpoint{2pt}{2.5pt}}
  \pgfpathlineto{\pgfqpoint{-4pt}{2.5pt}}
  \pgfpathlineto{\pgfqpoint{2pt}{-3.5pt}}
  \pgfpathclose
  \pgfusepathqfillstroke
}
\pgfdeclareplotmark{tsw}{
  \pgfpathmoveto{\pgfqpoint{-2pt}{-2.5pt}}
  \pgfpathlineto{\pgfqpoint{4pt}{-2.5pt}}
  \pgfpathlineto{\pgfqpoint{-2pt}{3.5pt}}
  \pgfpathclose
  \pgfusepathqfillstroke
}
\makeatother

\renewcommand{\arraystretch}{1.4}
\setlength{\tabcolsep}{4pt}
\definecolor{iridiumyellow}{HTML}{F3B207}
\definecolor{onewebblue}{HTML}{57BEF2}
%\renewcommand{\tabularxcolumn}[1]{m{#1}}

\begin{NiceTabularX}{\linewidth}{ %
    | >{\setstretch{1}\raggedright\arraybackslash}m{12mm} %
    | >{\setstretch{1}\raggedright\arraybackslash}m{32mm} %
    | *{4}{>{\centering\arraybackslash}X[m]|} %
      M{9mm} %
    | M{11mm} | %
  }[color-inside]
  \hline
    \Th{ID} &
    \Th{Architectural Decision} &
    \Th{Option A} &
    \Th{Option B} &
    \Th{Option C} &
    \Th{Option D} &
    \Th{Trac.} &
    \Th{Status} \\
  \hline
    \row{AD-01}{func}{On-board Autonomous Target Recognition (ATR)}{None}{Task-Triggered}{Continuous}{-}{PSF-3, FR-007}{Open}
  \hline
    \row{AD-02}{f2f}{Data Path}{Direct to Ground}{LEO-LEO Internal Relay Mesh}{LEO-LEO 3rd Party Relay Mesh}{LEO-GEO External Relay}{PSF-5, FR-005, FR-006}{Open}
  \hline
    \row{AD-03}{func}{Information Product}{Full Image Only}{Snippet/ Thumbnail}{Metadata}{-}{PSF-4}{Open}
  \hline
    \row{AD-04.1}{form}{Uplink Communication}{Optical Laser}{X-Band RF}{\sout{Ka-Band}

    S-Band RF}{-}{FR-005, FR-006, NFR-02}{Updated}
  \hline
    \row{AD-04.2}{form}{ Downlink Communication}{Optical Laser}{X-Band RF}{Ka-Band RF}{None (external Relay Downlink)}{FR-005, FR-006, NFR-02}{Open}
  \hline
    \row{AD-04.3}{form}{Inter-satellite Communication}{Optical Laser}{X-Band RF}{Ka-Band RF}{None}{FR-005, FR-006, NFR-02}{Open}
  \hline
    \row{AD-05}{f2f}{Data Routing Protocol}{Static Pre-Scheduled}{Adaptive Dynamic Mesh}{Store and Forward}{-}{PSF-5, NFR-01, NFR-02}{Open}
  \hline
    \row{AD-06}{form}{Ground Segment Network}{Dedicated Sovereign Network}{Commercial GSaaS (\eg KSAT/AWS)}{Hybrid (Sovereign + Commercial)}{-}{PSF-5, FR-006, NFR-01}{New}
  \hline
\CodeAfter
  \tikz[remember picture, overlay]{

    \node[draw=black, xshift=1.6mm, yshift=-5mm, color=black, inner sep=0, minimum width=26mm, minimum height=10mm, anchor=north east] (l) at (9-8.south east) {};

    \node[align=right,text width=42mm,anchor=north east] at (l.north west) {
      \RaggedLeft
      Legend for past design \\
      solutions described in \ref{sec:past-designs}:
    };

    % Iridium
    \draw[
      iridiumyellow,
      only marks,
      yshift=0mm,
      line width=0.8pt,
      mark=tnw,
      mark options={solid},
    ] plot coordinates {
      ([yshift=0mm]2-3.north west) % None
      ([yshift=0mm]3-4.north west) % LEO-LEO
      ([xshift=0mm]4-6.north west) % n/a (voice)
      ([yshift=0.6em]5-6.north west) % n/a (Ka-Band)
      ([yshift=0.6em]6-5.north west) % Ka-Band
      ([yshift=0.6em]7-5.north west) % Ka-Band
      ([yshift=0mm]8-4.north west) % Mesh
      ([yshift=0mm]9-3.north west) % Dedicated
      ([xshift=1.6mm,yshift=-1.6mm]l.north west)
    };
    \node[yshift=0.2em,xshift=-0.1em,below right,iridiumyellow!70!Sienna4,font=\tiny] at (4-6.north west) {Data (voice)};
    \node[yshift=0.8em,xshift=-0.1em,below right,iridiumyellow!70!Sienna4,font=\tiny] at (5-6.north west) {Ka-Band RF};
    \node[yshift=-0.4em,xshift=0.4em,below right,iridiumyellow!70!Sienna4,font=\footnotesize] at (l.north west) {Iridium};

    % OneWeb
    \draw[
      onewebblue,
      only marks,
      yshift=0mm,
      line width=0.8pt,
      mark=tne,
      mark options={solid},
    ] plot coordinates {
      ([yshift=0mm]2-3.north east) % None
      ([yshift=0.6em]3-3.north east) % Direct to ground
      ([xshift=0mm]4-6.north east) % n/a (data)
      ([yshift=0.6em]5-6.north east) % Ka-Band
      ([yshift=0.6em]6-5.north east) % Ka-Band
      ([yshift=0.6em]7-6.north east) % None
      ([yshift=0mm]8-3.north east) % Fixed
      ([yshift=0mm]9-5.north east) % Mixed
      ([xshift=-1.6mm,yshift=-1.6mm]l.north east)
    };
    \node[yshift=0.2em,xshift=-0.1em,below left,onewebblue!70!Turquoise4,font=\tiny] at (4-6.north east) {Data (internet)};
    \node[yshift=0.8em,xshift=-0.1em,below left,onewebblue!70!Turquoise4,font=\tiny] at (5-6.north east) {Ka-Band RF};
    \node[yshift=-0.4em,xshift=-0.4em,below left,onewebblue!70!Turquoise4,font=\footnotesize] at (l.north east) {OneWeb};

    % Starlink
    \draw[
      red,
      yshift=0mm,
      only marks,
      line width=0.8pt,
      mark=tsw,
      mark options={solid},
    ] plot coordinates {
      ([yshift=0mm]2-3.south west) % None
      ([yshift=0mm]3-4.south west) % LEO-LEO
      ([yshift=0mm]4-6.south west) % n/a (data)
      ([yshift=-0.6em]5-6.south west) % Ka-Band
      ([yshift=-0.6em]6-5.south west) % Ka-Band
      ([yshift=-0.6em]7-3.south west) % Laser
      ([yshift=0mm]8-4.south west) % Mesh
      ([yshift=0mm]9-3.south west) % Dedicated
      ([xshift=1.6mm,yshift=1.6mm]l.south west)
    };
    \node[yshift=-0.3em,xshift=-0.1em,above right,red,font=\tiny] at (4-6.south west) {Data (internet)};
    \node[yshift=-0.8em,xshift=-0.1em,above right,red,font=\tiny] at (5-6.south west) {Ka-Band RF};
    \node[yshift=0.4em,xshift=0.4em,above right,red,font=\footnotesize] at (l.south west) {Starlink};

    % % Iridium
    % \draw[
    %   iridiumyellow,
    %   only marks,
    %   yshift=0mm,
    %   line width=0.8pt,
    %   mark=*,
    %   mark size=1.8pt,
    %   mark options={solid},
    % ] plot coordinates {
    %   ([yshift=0mm]2-3.south west) % None
    %   ([yshift=0mm]3-4.south west) % LEO-LEO
    %   ([xshift=4mm]4-6.south west) % n/a (voice)
    %   ([xshift=4mm]5-6.north west) % n/a (Ka-Band)
    %   ([yshift=0mm]6-5.north west) % Ka-Band
    %   ([yshift=0mm]7-5.north west) % Ka-Band
    %   ([yshift=-1mm]8-4.west) % Mesh
    %   ([yshift=0mm]9-3.north) % Dedicated
    % };
    % \node[xshift=4mm,right,iridiumyellow!70!Sienna4,font=\tiny] at (4-6.south west) {Data (voice)};
    % \node[xshift=4mm,right,iridiumyellow!70!Sienna4,font=\tiny] at (5-6.north west) {Ka-Band RF};

    % % OneWeb
    % \draw[
    %   onewebblue,
    %   only marks,
    %   yshift=0mm,
    %   line width=0.8pt,
    %   mark=*,
    %   mark size=1.8pt,
    %   mark options={solid},
    % ] plot coordinates {
    %   ([yshift=0mm]2-3.south east) % None
    %   ([yshift=1.8mm]3-3.east) % Direct to ground
    %   ([xshift=0mm]4-6.north west) % n/a (data)
    %   ([xshift=0mm]5-6.south west) % Ka-Band
    %   ([yshift=0mm]6-5.south east) % Ka-Band
    %   ([yshift=0mm]7-6.west) % None
    %   ([yshift=0mm]8-3.south east) % Fixed
    %   ([yshift=0mm]9-5.south west) % Mixed
    % };
    % \node[yshift=0mm,xshift=0mm,right,onewebblue!70!Turquoise4,font=\tiny] at (4-6.north west) {Data (internet)};
    % \node[xshift=0mm,right,onewebblue!70!Turquoise4,font=\tiny] at (5-6.south west) {Ka-Band RF};

    % % Starlink
    % \draw[
    %   red,
    %   yshift=0mm,
    %   only marks,
    %   line width=0.8pt,
    %   mark=*,
    %   mark size=1.8pt,
    %   mark options={solid},
    % ] plot coordinates {
    %   ([yshift=0mm]2-3.north east) % None
    %   ([yshift=0mm]3-4.north east) % LEO-LEO
    %   ([yshift=0mm]4-6.north east) % n/a (data)
    %   ([yshift=-1.8mm]5-6.south east) % Ka-Band
    %   ([yshift=0mm]6-5.north east) % Ka-Band
    %   ([yshift=0mm]7-3.east) % Laser
    %   ([yshift=0mm]8-4.north west) % Mesh
    %   ([yshift=0mm]9-3.north west) % Dedicated
    % };
    % \node[yshift=0mm,xshift=0mm,below left,red,font=\tiny] at (4-6.north east) {Data (internet)};
    % \node[yshift=-1.8mm,above left,red,font=\tiny] at (5-6.south east) {Ka-Band RF};


  }%
  \vspace{10mm}
\end{NiceTabularX}

\end{table}

\vspace{-0.5em}

\noindent The following justifies how each of these architectural decisions collectively satisfy our problem space:
\begin{itemize}
  \item \textbf{Data Volume Reduction (AD-01, AD-03):} These ADs satisfy the ``low-latency'' requirement by minimizing the number of bits transmitted, thereby bypassing bandwidth bottlenecks.
  \item \textbf{Network Optimization (AD-02, AD-05, AD-06):} These ADs enable ``bypassing terrestrial infrastructure limitations'' and relative satellite positions by using an adaptive mesh to find the fastest path to the end-user.
  \item \textbf{Persistent Decision Flow (AD 04.1-04.3):} Defines the communication architecture that enables distributed routing and limits ground bottlenecks, directly supporting persistent, low-latency performance.
\end{itemize}

\noindent The identified architectural decisions (AD-01 through AD-06) represent a bounded and mutually exclusive set of technical tradeoffs.
Each row in \ref{tbl:decisions} forces a structural commitment that determines how information flows through the system.
No additional architectural lever meaningfully influences latency without falling into implementation detail.
Therefore, this set is sufficient to bound the tradespace while avoiding premature technology commitment.
To achieve low-latency, the architecture allows for trading on-board processing levels (AD-01, AD-03) to minimize the volume of data to be transmitted.
To bypass terrestrial limitations, the tradespace includes multiple networking and routing configurations (AD-02, AD-05) that enable the system to optimize the path to the end-user, regardless of satellite location.
Collectively, these open decisions provide the necessary degrees of freedom to converge on a design that satisfies the rigorous latency and persistence requirements of the project.
Therefore, at this stage, most decisions remain intentionally open to preserve architectural flexibility while bounding the tradespace with structurally meaningful options.
